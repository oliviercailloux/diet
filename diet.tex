\RequirePackage[l2tabu, orthodox]{nag}
\documentclass[version=3.21, pagesize, twoside=off, bibliography=totoc, DIV=calc, fontsize=12pt, a4paper]{scrartcl}
\input{preamble/packages}
\input{preamble/redac}
\input{preamble/math_basics}
\input{preamble/math_mine}

%I find these settings useful in draft mode. Should be removed for final versions.
	%Which line breaks are chosen: accept worse lines, therefore reducing risk of overfull lines. Default = 200.
		\tolerance=2000
	%Accept overfull hbox up to...
		\hfuzz=2cm
	%Reduces verbosity about the bad line breaks.
		\hbadness 5000
	%Reduces verbosity about the underful vboxes.
		\vbadness=1300

\title{Deliberated diet}
\author[1]{Olivier Cailloux}
\author[1]{Yasmine Balima}
\author[2]{Nicolas Salliou}
\author[1]{Yves Meinard}
\affil[1]{Université Paris-Dauphine, PSL Research University, CNRS, LAMSADE, 75016 PARIS, FRANCE}
\affil[2]{Institute for Spatial and Landscape Development, Planning of Landscape and Urban Systems (PLUS), ETH Zürich – Switzerland}
\hypersetup{
	pdfsubject={},
	pdfkeywords={},
}
\begin{document}
\maketitle

\section{Context}
This section presents the context of the study.
 
\subsection{Goal}
We aim at studying how we can capture the \ac{DJ} of an individual towards a complex, practical topic. The \ac{DJ} \citep{cailloux_formal_2020} of an individual towards a topic is the stance she adopts after careful examination of all relevant arguments concerning that topic. The examination is considered complete when the adopted stance is stable considering further counter-arguments.

In this exploratory study, we want to try to capture \acp{DJ} of many individuals on a particular question by confronting them to many arguments, and extract from this lessons about how to capture \acp{DJ} more generally, and about the difficulties that such a process gets confronted to. 

We proceed as follows. We choose a practical question as our topic. We select two champions, that is, two persons that know the arguments for and against particular stances on that question, and that have contrasting positions on which stance should be the deliberated one. In the \emph{collection} phase, the two champions defend their respective positions by recording their arguments for their position and against other positions in videos. In the \emph{adjudication} phase, individuals watch these videos, following a well-defined protocol, through a website. The individual is called a visitor (of the website). The visitor is questioned about the evolution of her \ac{DJ} along the process of watching the arguments and counter-arguments displayed in the videos.

\subsection{Topic}
\label{sec:topic}
We are interested in capturing the \acp{DJ} of individuals concerning a practical question in a fictitious setting. Here is the question we chose.

A local authority considers building a new public cantine. To what extend should the cantine be vegan?

To answer this question, our visitor will be able to adjust sliders on these three aspects (or something functionally equivalent, such as adjusting on two and observing the result on the third one):
\begin{itemize}
	\item number of days per week where only vegan options are offered
	\item number of days per week where only non-vegan options are offered
	\item number of days per week where both vegan and non-vegan options are offered
\end{itemize}
The sum of these three numbers must equal five.

\section{Collection phase}
From now on, assume we have two champions, $C_a$ and $C_b$.

We ask $C_a$ and $C_b$ for their favorite propositions, defined as $t_a$ and $t_b$ respectively as defined in \cref{sec:topic}). 

We ask $C_a$ to produce a video in favor of $t_a$, and label it $s_{1a}$. We send $s_{1a}$ to $C_b$. We give $C_b$ a choice: either produce a reply video (arguing for $t_b$ against the video arguing for $t_a$), or start a new thread in favor of $t_b$, or both. In the first case, we label the argument $s_{1ab}$. In the second case, we label the argument $s_{1b}$. The scheme is then repeated. We also run a parallel scheme starting with $b$ instead of $a$.

Here is a more precise description of the collection phase.

\begin{itemize}
	\item At the start of the whole procedure, both champions are fully informed about the procedure and the future use of their arguments, thanks to discussions and through sending them a copy of this whole document.
	\item The naming scheme is such that the last letter of a video indicate its author.
	\item Define $\text{init}(\alpha)$, with $\alpha \in \set{a, b}$, as follows. Precondition: $C_\alpha$ has received no argument of any sort from her opponent. We ask $C_\alpha$ to produce a video in favor of $t_\alpha$, and label the resulting video $s_{1 \alpha}$.
	\item Define $\text{next}(\alpha)$, with $\alpha \in \set{a, b}$, as one plus the greatest integer numbering a start video recorded by $\alpha$ in the whole process so far. For example, if $\alpha = b$, and if a video labeled $s_{3b}$ exists already but no video labeled $s_{4b}$ exists yet, then $\text{next}(b) = 4$.
	\item Define $\text{reply}(S)$, with $S$ a set of previously produced videos by a given champion $\alpha$, as follows. We address the champion $\beta$, with $\beta ≠ \alpha$. We send her the videos in $S$. We invite her to reply to every videos she wants to reply to by recording new videos, in about two weeks time. A video replying to another video is labeled by suffixing $\beta$ to the label of the video it replies to. For example, if $\alpha = a$, when replying to a video labeled $s_{1aba}$, we label it $s_{1abab}$. The set of replies may be empty. She may also start a new video that is not a reply, that we name $s_{k\beta}$, with $k$ equal to $\text{next}(\beta)$.
	\item We start with $\text{init}(a)$ and $\text{init}(b)$ in parallel, therefore obtaining $s_{1a}$ and $s_{1b}$. Define $S_{1a} = \set{s_{1a}}$ and $S_{1b} = \set{s_{1b}}$.
	\item After having obtained a set of videos $S$, if $S ≠ \emptyset$, we run $\text{reply}(S)$, and label the set of resulting videos by suffixing the identifier of their author to the label of $S$. 
	\item We repeat the previous step, running two threads in parallel whenever possible, until both participants think that all arguments that is important to form a \ac{DJ} has been given.
	\item We ask the champions to label, in maximum 30 characters, each of their own videos and select an image from the video that becomes its “thumbnail”.
\end{itemize}

\begin{example}
Here is an example run.
\begin{enumerate}
	\item We start with $\text{init}(a)$ and $\text{init}(b)$ in parallel and obtain $s_{1a}$ and $s_{1b}$. Define $S_{a} = \set{s_{1a}}$ and $S_{b} = \set{s_{1b}}$.
	\item We apply $\text{reply}(\set{s_{1a}})$ and obtain $S_{ab} = \set{s_{1ab}, s_{2b}}$.
	\item In parallel, we apply $\text{reply}(\set{s_{1b}})$ and obtain $S_{ba} = \set{s_{1ba}}$.
	\item We apply $\text{reply}(S_{ab})$ (as soon as $S_{ab}$ is available) and obtain $S_{aba} = \set{s_{1aba}, s_{2ba}, s_{2a}}$.
	\item We apply $\text{reply}(S_{ba})$ (as soon as $S_{ba}$ is available), but $C_b$ sees no need to answer those arguments; we obtain $S_{bab} = \emptyset$.
	\item We apply $\text{reply}(S_{aba})$ (as soon as $S_{aba}$ is available), where $C_b$ sees an opportunity for answering; we obtain $S_{abab} = \set{s_{2bab}}$.
	\item We apply $\text{reply}(S_{abab})$ (as soon as the argument is available), $C_a$ sees no need of counter-arguing, and we obtain $S_{ababa} = \emptyset$.
\end{enumerate}
\end{example}

\section{Adjudication phase}
When the collection phase is over, we put every collected video on a web site and design everything so that the adjudication phase of the protocol can be run, as described in this section. This phase consists, for each visitor, in the following.

An individual $i$ comes to our web site. This visitor has to go through the \emph{context presentation} and can then proceed to the \emph{deliberation} part.

\subsection{Context presentation}
In the context presentation part, the visitor is explained the context of the study by written text. The following points should be made clear to the visitor.
\begin{itemize}
	\item Introduce the context by explaining, for example, that Berlin has introduced a vegan canteen for its student (cite \hrefblue{https://www.theguardian.com/world/2019/apr/25/the-need-was-there-berlins-first-vegan-canteen-for-students-opens}{article}). Explain that we (fictitiously) wonder if we should do it as well in some (unspecified?) city in France.
	\item We (the team designing the study) are interested in knowing what the visitor considers are the best arguments to form a \ac{DJ} on this topic, as judged by himself. Our long-term goal is to help individuals form a well-informed opinion by displaying to them arguments that tend to be considered good by many individuals.
%	\item We have no conflict of interest and do not try to promote either diet option (contrary to our champions). ⇒ We decided not to write this as is.	\commentOC{It will strike the visitor that we seem biased towards veganism or vegetarianism: no champion defend a meat-extensive diet. What should we say about this?}
	\item Two well-known public figures have argued for two specific options. Those champions, contrary to us, actively try to promote one of the options.
%	\item He is asked whether he is willing to adopt an open mind, that is, to be ready to change his mind \commentOC{change opinion} if given arguments that he considers good for a side that he does not consider a priori as the “right” choice. \commentOC{We decided that this can be naïve or counter-productive.}
%	\item He should spend 45 minutes, (mostly) uninterrupted, on this experiment. During this time, he will have to look at videos from both sides, with a balanced time for both champions, and answer questions about his current \ac{DJ} and about which arguments he finds good \commentOC{We decided not to because we have to ask the questionaire regularly anyway: we are on the internet.} 
%	\item After this time, he is free to continue exploring the videos and answering further questions in an unconstrained way.
	\item The visitor is asked to agree to our use of the collected data, in order to be GDPR-compliant.
	\item The visitor is shown the question of the topic and the possible choices, and is asked about his initial opinion.
\end{itemize}
Only after the visitor has accepted those conditions can he start the deliberation part. 

\subsection{Deliberation}
During the deliberation part, the visitor can still access the explanations relating to the context presentation but it is not displayed prominently any more as we expect it will not be her main interest in that phase.

The constrained deliberation phase is composed of \textbf{video steps} and \textbf{questionaire steps}. 

\subsubsection{Video step}
Here is the description of a video step. In summary, a video step consists in letting the visitor choose a video among those she has not watched yet, and letting her watch it (to the end or not), and decide to mark it as “seen”; then answering a question about the video.
\begin{itemize}
	\item Each video step starts with an associated set of videos “proposed”, $S_P$, and an associated list of videos “seen”, $S_S$. The list $S_S$ is possibly augmented at the end of each step while the set $S_P$ is computed at the start of each step from the list $S_S$ resulting from the previous step. Two time counters $\theta_a$ and $\theta_b$ keep track of the time spent so far watching videos of each champion.
	\item At the start, $S_S = \emptyset$; $\theta_a = 0$; $\theta_b = 0$.
	\item A “video step” consists in the set $S_P$ being shown to the individual among which he can choose the video he will watch during this step; and he can also watch again videos that are “seen”. Each video is displayed as its thumbnail, with its label shown clearly, in a randomized order; the videos from $S_S$ are clearly distinguished and displayed afterwards, in the order they have entered the list (the order they have been marked as “seen”). He clicks on a video and starts watching it. 
He can navigate in the timeline of the video (a la youtube).
The step is finished with one of these possibilities.
	\begin{itemize}
		\item If the video is watched until the end with no navigation in the timeline, in which case the video is added at the end of $S_S$. 
		\item The user can also click to mark the video as seen without having watched it entirely, in which case the video is also added at the end of $S_S$.
		\item The user can also click to stop the video without marking it as seen; in which case $S_S$ is not modified.
	\end{itemize}
	\item Given a video $s$, define $r(s)$ as the singleton set containing the reply video to $s$ (thus produced by the other champion than the author of $s$), if such a video exists, and $\emptyset$ otherwise. For example, $r(s_{2aba}) = \set{s_{2abab}}$ if such a video exists. At the start of a step where the list of videos seen is $S_S$, $S_N$ (for “next”) is defined as the non-seen videos among the starting videos and the videos replying to a video that has been seen, thus, $S_N = \left(\set{s_{k\alpha}, k \in \N, \alpha \in \set{a, b}} \cup \bigcup_{s \in S_S} r(s)\right) \setminus S_S$. The proposed videos in this step are defined as $S_P = S_N$; \emph{except} if the time allowed to watch one of the champions has been exhausted. The visitor has 40 minutes max to spend for each champion. Thus, if $\theta_a ≥ 40 \text{ minutes}$, $S_P$ consists in the videos of champion $b$ among $S_N$.
	\item At any time (except when watching a video in full-screen), $i$ sees how much time he has spent watching videos of each champion.
	\item At the end of the video step, the visitor is asked: did you find this video helpful to form a \ac{DJ} about this question? Yes (green check mark); Undecided (question mark); No (red cross). The default answer (that is kept if the visitor does not answer the question) is Undecided.
\end{itemize}

\subsubsection{Questionaire step}
We take the visitor to a questionaire step in between some videos step. The first questionaire step asks the visitor some classical socio-demographic questions. The second one asks about her openness. The third one asks whether she is willing to leave some coordinates to us so that she can be contacted in a few months to ask about her experience and \ac{DJ} (she can leave an e-mail address and a phone number). We also take the opportunity of the questionaire steps to give feedback, such as the fraction of arguments seen; the time spent for each champion; a graph of the evolution of his opinion; … 

We also ask the following questions to $i$ (not necessarily in each of the questionaire steps).
\begin{itemize}
	\item Which answer would you choose if you had to choose now (referring to \cref{sec:topic})?
	\item Which videos did you personally find most helpful to form a \ac{DJ} about this question?
	\item Are there some arguments that you think have not been used from either champions and should have? (free text answer)
\end{itemize}
For the “which videos” question, $i$ is shown the list of thumbnails of videos in $S_S$, followed by the thumnails of videos he has partially seen but not marked as seen, together with the marks he has left previously (green check or red cross mark), if any, and can check any thumnail he wants, from both champions. 

\section{Analysis}
Here are some example items that we will want to analyze using the collected data.
\begin{itemize}
	\item Do visitors change opinion?
	\item Do visitors tend to spend more time watching videos from the side they favor a priori? Does this switch when they change their opinion? We count the total time spent watching videos from each champion. We count the time really spent playing a video, including replays when $i$ has watched several times the same (part of a) video, and thus not counting fully a video that has been partly watched, even if the video has been included in $S_S$ because of an explicit demand from $i$. This permits to count time allowed to each champion.
	\item Can we reasonably say that we have captured \acp{DJ}? Is the position stable compared to counter-arguments found using other sources (such as online debate web sites)? We could compare stability with and without the protocol by contacting again our visitors and other persons; or compare this protocol and another one.
	\item Do people agree on which videos are helpful? Try to cluster people so that they agree on this inside a cluster. Is this usable to shorten the time to deliberate?
\end{itemize}

% We agree that one champion can discard one of their videos during the process, but only if they both agree and in case of an ambiguity or other similar big problem.

\bibliography{biblio}
\end{document}

